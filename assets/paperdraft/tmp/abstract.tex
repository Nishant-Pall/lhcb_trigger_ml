The use of multivariate classifiers has become commonplace in particle physics.
To enhance the performance, a series of classifiers is typically trained; this is a technique known as boosting. 
This paper explores several novel boosting methods that have been designed to produce a uniform selection efficiency in a chosen multivariate space.  
Such algorithms have a wide range of applications in particle physics,
from producing uniform signal selection efficiency across a Dalitz-plot to avoiding the creation of false signal peaks in an invariant mass distribution when searching for new particles.
